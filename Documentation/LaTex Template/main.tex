% If you want 16:9 add [aspectratio=169] before {beamer}
% Otherwise by default, it is 4:3 
\documentclass[aspectratio=169]{beamer}%
% Closing navigation
\setbeamertemplate{navigation symbols}{}%
% Theme
\usetheme{Boadilla}%
% Code listings
\usepackage{listings}%
% UTF8
\usepackage[utf8]{inputenc}%
% Color pack
\usepackage{xcolor}%

% Enumeration and Itemization settings
\setbeamertemplate{itemize items}[square]
\setbeamertemplate{enumerate items}[default]%

% Beamer Colors
\definecolor{footblue}{RGB}{0,84,160}%
\definecolor{footaqua}{RGB}{0,138,203}%
\setbeamercolor{author in head/foot}{fg=white, bg=footaqua}%
\setbeamercolor{title in head/foot}{fg=white, bg=footblue}%
\setbeamercolor{date in head/foot}{fg=white, bg=footaqua}%
\setbeamercolor{frametitle}{fg=footblue, bg=white}%
\setbeamercolor{titlelike}{fg=footblue}%
\setbeamercolor{itemize item}{fg=footblue}%
\setbeamercolor{itemize subitem}{fg=footblue}%
\setbeamercolor{itemize subsubitem}{fg=footblue}%
\setbeamercolor{itemize subsubsubitem}{fg=footblue}%
\setbeamercolor{enumerate item}{fg=footblue}%
\setbeamercolor{enumerate subitem}{fg=footblue}%
\setbeamercolor{enumerate subsubitem}{fg=footblue}%
\setbeamercolor{enumerate subsubsubitem}{fg=footblue}%
\setbeamercolor{caption name}{fg=footblue}%

% Title settings
\title{ALGO-101}%
\subtitle{Week X - Lecture Name}%
\author{YOUR NAME}%
\date{MONTH YEAR}%
\institute{ITU ACM}%

% Footline settings
\defbeamertemplate*{footline}{Boadilla}%
{%
    \leavevmode%
    \hbox{%
        \begin{beamercolorbox}[wd=.225\paperwidth,ht=2.25ex,dp=1ex,center]{author in head/foot}%
            \usebeamerfont{author in head/foot}\insertshortauthor%
        \end{beamercolorbox}%
        \begin{beamercolorbox}[wd=.55\paperwidth,ht=2.25ex,dp=1ex,center]{title in head/foot}%
            \usebeamerfont{title in head/foot}\insertshorttitle {} \insertsubtitle
        \end{beamercolorbox}%
        \begin{beamercolorbox}[wd=.225\paperwidth,ht=2.25ex,dp=1ex,right]{date in head/foot}%
            \usebeamerfont{date in head/foot}\insertshortdate{}\hspace*{2em}%
            \insertframenumber{} / \inserttotalframenumber\hspace*{2ex}%
        \end{beamercolorbox}}%
    \vskip0pt%
}%

% Code highlight colors
\definecolor{codegreen}{rgb}{0,0.6,0}%
\definecolor{codered}{rgb}{0.6,0,0}%
\definecolor{codegray}{rgb}{0.5,0.5,0.5}%
\definecolor{codepurple}{rgb}{0.58,0,0.82}%
\definecolor{backcolour}{rgb}{0.95,0.95,0.92}%

% Code highlight settings
\lstdefinestyle{mystyle}%
{%
    commentstyle=\color{codegreen},%
    keywordstyle=\color{blue},%
    numberstyle=\tiny\color{codegray},%
    stringstyle=\color{codered},%
    basicstyle=\ttfamily\footnotesize,%
    breakatwhitespace=false,%
    breaklines=true,%
    captionpos=b,%
    keepspaces=true,%
    numbers=left,%
    numbersep=5pt,%
    showspaces=false,%
    showstringspaces=false,%
    showtabs=false,%
    tabsize=4,%
    xleftmargin=12pt,%
}%
% Applying code highlight style
\lstset{style=mystyle}%

\begin{document}%

% Creating title page
\frame{\titlepage}%

\begin{frame}{Topics}
    % Text here
    Topics covered at week 1:
    % Itemized List
    \begin{itemize}%
        \item Big O Notation
        \item Recursion
        \item CPP STL
            \begin{itemize}%
                \item Vector
                \item String
                \item Map
                \item Set
            \end{itemize}%
        \item Searching Algorithms
    \end{itemize}%
\end{frame}%

\begin{frame}{Topics Centered (Enumerated)}%
    \begin{columns}[t]%
        % column for centering text, adjust width accordingly
        \begin{column}{0.3\textwidth}% Column width
            % Text here
            Topics covered at week 1:
            \begin{enumerate}%
                \item Big O Notation
                \item Recursion
                \item CPP STL
                    \begin{enumerate}%
                        \item Vector
                        \item String
                        \item Map
                        \item Set
                    \end{enumerate}%
                \item Searching Algorithms
            \end{enumerate}%
        \end{column}%
    \end{columns}%
\end{frame}%

\begin{frame}{Sample Code}%

    % to import your code from code file write code file's name into the curly braces
    % if you want to show some specific lines use linerange
    % linerange={1-3, 7-8} ===> shows the 1, 2, 3, 7 and 8
    
    \lstinputlisting[language = C++]{deneme.cpp}%
\end{frame}%

\begin{frame}{Seperated Screen}%
    \begin{columns}[T] % align columns
        % To add a new column 
        % start with \begin{column}{.x\textwidth}
        % end with \end{column}
        % add \hfill between columns
        % be careful that sum of all column widths should not exceed .96\textwidth
        \begin{column}{.64\textwidth}% Column width here 
            % Edit left part here
             Left Part text
            \lstinputlisting[language = C++, linerange={1-13}]{deneme.cpp}%
        \end{column}%
        \hfill
        \begin{column}{.32\textwidth}% Column width here
            % Edit right part here
            Right Part text
            % Figure
            \begin{figure}[!ht]%
                \centering%
                \includegraphics[width=.7\linewidth]{acm_algo_logo.png}% Name of the picture file 
                \caption{Caption goes here}% Caption text, if you don't want, you can delete line \caption
            \end{figure}%
        \end{column}%
    \end{columns}
\end{frame}

\begin{frame}{Example to add pictures}
    Note that svg files are problematic in LaTex, therefore export svg images to png first, then add them as png. Use high resolution.
    % Figure 
    \begin{figure}[!ht]%
        \centering%
        \includegraphics[width=.3\textwidth]{acm_algo_logo.png}% Name of the picture file 
        \caption{Caption goes here}% Caption text, if you don't want, you can delete line \caption
    \end{figure}%
\end{frame}%

\begin{frame}{Example Table}%
    You can complete this table if you add \textbackslash hline command at the end and at \textbackslash begin \{ tabular \} there should be an extra vertical bar near r in the arguements
    \begin{table}[ht]% table align
        \centering%
        \begin{tabular}{|l|c|r}% | -> vertical lines c,l,r -> data, center, left, right aligned
            \hline % you can add lines with \hline
            1 & 2 & 3 \\ % data & data .... \\ & means spacer between data cells and \\ means start new line
            \hline % line between each row
            111 & 222 & 333 
        \end{tabular}%
        \caption{Caption}%
        \label{tab:my_label}%
    \end{table}%
\end{frame}%

\end{document}%
